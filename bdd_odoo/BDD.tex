\documentclass{article}
\usepackage{listings}
\usepackage{color}
\usepackage{hyperref}
\usepackage{geometry}
\geometry{margin=1in}

\definecolor{lightgray}{gray}{0.9}
\lstset{
  backgroundcolor=\color{lightgray},
  basicstyle=\ttfamily,
  breaklines=true
}

\title{BDD Deployment and Service Setup Report}
\author{}
\date{\today}

\begin{document}

\maketitle

\section{Introduction}
This report details the steps and configurations used to create a BDD (Business Driven Development) deployment and service in a Kubernetes cluster. It includes YAML configurations for deployment, services, persistent volume claims, and secrets, followed by the commands used to apply these configurations.

\section{Deployment Configuration}
The deployment configuration is as follows:

\begin{lstlisting}[language=yaml]
apiVersion: apps/v1
kind: Deployment
metadata:
  name: bdd-odoo-deployment
  namespace: icgroup
  labels:
    app: bdd-odoo
spec:
  replicas: 1
  selector:
    matchLabels:
      app: bdd-odoo
  template:
    metadata:
      labels:
        app: bdd-odoo
    spec:
      containers:
      - name: bdd-odoo
        image: postgres:latest
        ports:
        - containerPort: 5432
        env:
        - name: POSTGRES_DB
          value: odoo
        - name: POSTGRES_USER
          value: odoo
        - name: POSTGRES_PASSWORD
          valueFrom:
            secretKeyRef:
              name: odoo-db-secret
              key: password
        volumeMounts:
        - mountPath: /var/lib/postgresql/data
          name: bdd-odoo-storage
        resources:
            requests:
              memory: "256Mi"
              cpu: "500m"
            limits:
              memory: "512Mi"
              cpu: "1"
      volumes:
      - name: bdd-odoo-storage
        persistentVolumeClaim:
          claimName: bdd-odoo-pvc

\end{lstlisting}

\section{Service Configuration}
The service configuration is as follows:

\begin{lstlisting}[language=yaml]
apiVersion: v1
kind: Service
metadata:
  name: bdd-odoo-service
  namespace: icgroup
  labels:
    app: bdd-odoo
spec:
  selector:
    app: bdd-odoo
  ports:
    - protocol: TCP
      port: 5432
      targetPort: 5432
  type: ClusterIP

\end{lstlisting}

\section{Persistent Volume Claim (PVC) Configuration}
The PVC configuration is as follows:

\begin{lstlisting}[language=yaml]
apiVersion: v1
kind: PersistentVolumeClaim
metadata:
  name: bdd-odoo-pvc
  namespace: icgroup
spec:
  accessModes:
    - ReadWriteOnce
  resources:
    requests:
      storage: 10Gi

\end{lstlisting}

\section{Secret Configuration}
The secret configuration is as follows:

\begin{lstlisting}[language=yaml]
apiVersion: v1
kind: Secret
metadata:
  name: odoo-db-secret
  namespace: icgroup
type: Opaque
data:
  password: cGFzc3dvcmQ=  # "password" en base64
\end{lstlisting}

\section{Applying the Configurations}
To apply the above configurations to your Kubernetes cluster, use the following commands:

\begin{verbatim}
kubectl apply -f secret.yaml
kubectl apply -f pvc.yaml
kubectl apply -f deployment.yaml
kubectl apply -f cluster-IP-service.yaml
\end{verbatim}

\section{Conclusion}
This report covered the necessary steps and configurations required to set up a BDD deployment and service in a Kubernetes environment. By following the steps and applying the provided configurations, the BDD deployment should be successfully created and operational.

\end{document}
