\documentclass{article}
\usepackage{listings}
\usepackage{xcolor}
\usepackage{hyperref}

\hypersetup{
    colorlinks=true,
    linkcolor=blue,
    filecolor=magenta,      
    urlcolor=blue,
    pdftitle={Kubernetes Deployment Setup},
    pdfpagemode=FullScreen,
}

\lstdefinelanguage{yaml}{
    keywords={true,false,null,y,n},
    keywordstyle=\color{blue},
    basicstyle=\ttfamily\small,
    comment=[l]{\#},
    morecomment=[s]{/*}{*/},
    commentstyle=\color{gray},
    stringstyle=\color{orange},
    moredelim=[l][\color{green}]{---},
    moredelim=[l][\color{green}]{...},
    morestring=[b]',
    morestring=[b]"
}

\lstset{
    basicstyle=\ttfamily,
    keywordstyle=\color{blue},
    commentstyle=\color{gray},
    stringstyle=\color{orange},
    showstringspaces=false,
    breaklines=true,
    frame=single,
}

\begin{document}

\title{Kubernetes Deployment Setup for \texttt{ic-webapp}}
\author{}
\date{}
\maketitle

\section*{Introduction}
This document outlines the steps to set up and deploy the \texttt{ic-webapp} application on a Kubernetes cluster using Minikube with Docker as the driver.

\section*{Prerequisites}
\begin{itemize}
    \item \textbf{Docker}: Ensure Docker is installed and running.
    \item \textbf{Minikube}: Installed and configured to use Docker driver.
    \item \textbf{kubectl}: Command-line tool for interacting with the Kubernetes cluster.
\end{itemize}

\section*{Installation Steps}

\subsection*{Installing Docker}
1. Download Docker Desktop from the \href{https://www.docker.com/products/docker-desktop}{Docker website}.
2. Follow the installation instructions on the website.
3. After installation, ensure Docker is running.

\subsection*{Installing Minikube}
1. Download the Minikube installer for Windows from the \href{https://minikube.sigs.k8s.io/docs/start/}{Minikube website}.
2. Install Minikube by running the installer.
3. Ensure Minikube uses Docker as the driver:
   \begin{lstlisting}[language=bash]
   minikube config set driver docker
   \end{lstlisting}

\subsection*{Installing kubectl}
1. Download the kubectl binary for Windows from the \href{https://kubernetes.io/docs/tasks/tools/install-kubectl-windows/}{Kubernetes website}.
2. Add the binary to your system's PATH by following the installation instructions on the website.
3. Verify the installation:
   \begin{lstlisting}[language=bash]
   kubectl version --client
   \end{lstlisting}

\section*{Setup Steps}

\subsection*{1. Start Minikube}
\begin{lstlisting}[language=bash]
minikube start --driver=docker --kubernetes-version=v1.20.0
\end{lstlisting}

\subsection*{2. Create Namespace}
Create a namespace for organizing resources.
\begin{lstlisting}[language=yaml]
# namespace.yaml
apiVersion: v1
kind: Namespace
metadata:
  name: icgroup
\end{lstlisting}

Apply the namespace:
\begin{lstlisting}[language=bash]
kubectl apply -f namespace.yaml
\end{lstlisting}

\subsection*{3. Create Deployment}
Define the deployment for \texttt{ic-webapp}.
\begin{lstlisting}[language=yaml]
# deployment.yaml
apiVersion: apps/v1
kind: Deployment
metadata:
  name: ic-webapp
  namespace: icgroup
  labels:
    env: prod
spec:
  replicas: 2
  selector:
    matchLabels:
      app: ic-webapp
  template:
    metadata:
      labels:
        app: ic-webapp
    spec:
      containers:
        - name: ic-webapp
          image: imane123456788/file_rouge:v1
          ports:
            - containerPort: 8080
          resources:
            requests:
              memory: "256Mi"
              cpu: "500m"
            limits:
              memory: "512Mi"
              cpu: "1"
\end{lstlisting}

Apply the deployment:
\begin{lstlisting}[language=bash]
kubectl apply -f deployment.yaml
\end{lstlisting}

\subsection*{4. Create Services}

\subsubsection*{NodePort Service}
Expose the deployment using NodePort.
\begin{lstlisting}[language=yaml]
# nodeport-service.yaml
apiVersion: v1
kind: Service
metadata:
  name: ic-webapp-nodeport
  namespace: icgroup
spec:
  selector:
    app: ic-webapp
  ports:
    - protocol: TCP
      port: 80
      targetPort: 8080
      nodePort: 30007
  type: NodePort
\end{lstlisting}

Apply the service:
\begin{lstlisting}[language=bash]
kubectl apply -f nodeport-service.yaml
\end{lstlisting}

\subsubsection*{LoadBalancer Service}
Optionally, expose the deployment using LoadBalancer.
\begin{lstlisting}[language=yaml]
# loadbalancer-service.yaml
apiVersion: v1
kind: Service
metadata:
  name: ic-webapp-loadbalancer
  namespace: icgroup
spec:
  selector:
    app: ic-webapp
  ports:
    - protocol: TCP
      port: 80
      targetPort: 8080
  type: LoadBalancer
\end{lstlisting}

Apply the service:
\begin{lstlisting}[language=bash]
kubectl apply -f loadbalancer-service.yaml
\end{lstlisting}

\subsection*{5. Verify Services}
Check the status of the services:
\begin{lstlisting}[language=bash]
kubectl get services -n icgroup
\end{lstlisting}

\subsection*{6. Access the Application}

\subsubsection*{Using NodePort}
Find the Minikube IP:
\begin{lstlisting}[language=bash]
minikube ip
\end{lstlisting}

Access the application at \texttt{http://<minikube-ip>:30007}.

\subsubsection*{Using Minikube Service Command}
\begin{lstlisting}[language=bash]
minikube service ic-webapp-nodeport -n icgroup --url
\end{lstlisting}

\subsubsection*{Using Port Forwarding}
\begin{lstlisting}[language=bash]
kubectl port-forward svc/ic-webapp-nodeport 8080:80 -n icgroup
\end{lstlisting}

Access the application at \texttt{http://localhost:8080}.

\section*{Troubleshooting}

\subsection*{External IP Pending}
If the external IP for LoadBalancer is pending, use NodePort or \texttt{minikube service} command.

\subsection*{Cannot Access NodePort}
Ensure Docker for Windows networking is configured correctly. Use port forwarding as an alternative.

\section*{References}
\begin{itemize}
    \item \href{https://minikube.sigs.k8s.io/docs/}{Minikube Documentation}
    \item \href{https://kubernetes.io/docs/}{Kubernetes Documentation}
    \item \href{https://docs.docker.com/}{Docker Documentation}
\end{itemize}

\end{document}
